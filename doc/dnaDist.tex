\documentclass[a4paper, english]{article}
\usepackage{graphics,eurosym,latexsym}
\usepackage{listings}
\lstset{columns=fixed,basicstyle=\ttfamily,numbers=left,numberstyle=\tiny,stepnumber=5,breaklines=true}
\usepackage{pst-all}
\usepackage{algorithmic,algorithm}
\usepackage{times}
\usepackage{babel}
\usepackage[nodayofweek]{datetime}
\usepackage[round]{natbib}
\newcommand{\ty}{\texttt}
\bibliographystyle{plainnat}
\oddsidemargin=0cm
\evensidemargin=0cm
\textwidth=16cm
\textheight=23cm
\begin{document}

\title{\texttt{dnaDist} \input{version}: Compute Evolutionary
  Distances between DNA Sequences}
\author{Bernhard Haubold\ignorespaces
}
\input{date}
\date{\displaydate{tagDate}}
\maketitle

\section{Introduction} 
Given a set of aligned DNA sequences in FASTA format, \ty{dnaDist}
computes the evolutionary distance between all pairs. There is a
choice between two types of distances, raw and the default
Jukes-Cantor. Raw distances, $d$, are the number of differences between two
sequences divided by the number of nucleotides compared. The
Jukes-Cantor distance, $K$, corrects this raw distance for
multiple mutations \citep{juk69:evo}:
\[
K=-\frac{3}{4}\log\left(1-\frac{4}{3}d\right).
\]
$K$ is also known as the number of substitutions per site and is
always greater or equal to $d$. The output of \ty{dnaDist} is in
PHYLIP format \citep{fel05:phy}.

In addition to straight forward distance computation, \ty{dnaDist}
also implements two types of bootstrap analysis. The ``classical''
bootstrap by \cite{fel85:con} and the pairwise bootstrap
by \cite{klo16:sup}. 

\section{Getting Started}
\texttt{DnaDist} was written in C on a computer running Linux.
Please contact \texttt{haubold@evolbio.mpg.de\ignorespaces
} if there are any problems
with the program.
\begin{itemize}
\item Obtain the package\\
\texttt{git clone https://www.github.com/evolbioinf/dnaDist}
\item Change into the directory just downloaded
\begin{verbatim}
cd dnaDist
\end{verbatim}
and make \texttt{dnaDist}
\begin{verbatim}
make
\end{verbatim}
\item Test \texttt{dnaDist}
\begin{verbatim}
make test
\end{verbatim}
\item The executable \texttt{dnaDist} is located in the
  directory \texttt{build}. Place it into your \texttt{PATH}.
\item Make the documentation
\begin{verbatim}
make doc
\end{verbatim}
This calls the typesetting program \texttt{latex}, so please make sure
it is installed before making the documentation. The typeset manual is
located in
\begin{verbatim}
doc/dnaDist.pdf
\end{verbatim}
\end{itemize}

\section{Tutorial}
\begin{itemize}
\item Run the default distance estimation to compute the number of
  substitutions per site
\begin{verbatim}
dnaDist data/test.fa 
2
S1        0.000000 0.010067
S2        0.010067 0.000000
\end{verbatim}
\item To get the underlying number of mismatches
  per site, use \ty{-r 1} switch
\begin{verbatim}
dnaDist -r 1 data/test.fa 
2
S1        0.000000 0.010000
S2        0.010000 0.000000
\end{verbatim}
Notice that the number of mismatches is slightly smaller than the
number of substitutions.
\item Similarly, to get the total number of mismatches,
\begin{verbatim}
./build/dnaDist -r 2 data/test.fa 
2
S1        0 10
S2        10 0
\end{verbatim}
\item Compute distances from a single bootstrap replicate of the input alignment \citep{fel85:con}
\begin{verbatim}
dnaDist -s 13 -b 1 data/test.fa 
2
S1        0.000000 0.014132
S2        0.014132 0.000000
\end{verbatim}
The \ty{-s} option sets the seed for the random number generator to
ensure numerically identical results for testing. By default, the seed
is generated internally to ensure independence of runs. Also, in a
proper bootstrap analysis the number of replicates would be, say,
$10^4$.
\item We can also get the uncorrected raw distance for the bootstrap
  replicate:
\begin{verbatim}
dnaDist -r -s 13 -b 1 data/test.fa 
2
S1        0.000000 0.010000
S2        0.010000 0.000000
\end{verbatim}
\item Compute one set of distances with pairwise bootstrapping
  \citep{klo16:sup}:
\begin{verbatim}
dnaDist -s 13 -p 1 data/test.fa 
2
S1        0.000000 0.012097
S2        0.012097 0.000000
\end{verbatim}
\item And again, \ty{-r 1} gives us the slightly smaller corresponding
  uncorrected result:
\begin{verbatim}
dnaDist -r 1 -s 13 -p 1 data/test.fa 
2
S1        0.000000 0.011000
S2        0.011000 0.000000
\end{verbatim}
\end{itemize}

\section{Change Log}
Please use
\begin{verbatim}
git log
\end{verbatim}
to list the change history.

\bibliography{ref}
\end{document}

